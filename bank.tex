\documentclass[letterpaper,twoside]{article}

% Preamble {{{
\usepackage[margin=1in]{geometry}
\usepackage[dvipsnames]{xcolor}
\usepackage{amsmath, amsfonts, amssymb}
\usepackage{amsthm}
\usepackage{thmtools}
\usepackage{txfonts,pxfonts}
\usepackage[many]{tcolorbox}
\usepackage{wasysym}
\usepackage{hyperref}
\usepackage{enumerate}
\usepackage{pgfplots}
\usepackage{fancyhdr}
\usepackage{lastpage}
\usepackage{harpoon}

\usepackage{lipsum}
\usepackage{cancel}
\usepackage[symbol]{footmisc} % replaces footnotes with things that aren't letters.


\renewcommand{\thempfootnote}{\thefootnote}
\renewcommand{\proofname}{Begin Proof}
\pgfplotsset{compat=newest}
\renewcommand{\qedsymbol}{\hfill\textcolor{pfsd}{$\blacksquare$}}

\newcommand{\N}{\mathbb{N}}
\newcommand{\R}{\mathbb{R}}
\newcommand{\Q}{\mathbb{Q}}

\newcounter{ex}
\definecolor{solsd}{RGB}{51,0,111}
\definecolor{dubg}{RGB}{232,227,211}
\newtcolorbox[use counter=ex]{que}[1][Question \thetcbcounter]{
  colback=dubg, colframe=solsd,
  fonttitle=\bfseries, enhanced,
  attach boxed title to top left, boxed title style={sharp corners, colback=solsd}, boxrule=0pt, leftrule=4pt, rightrule=2pt,
sharp corners, breakable, title={#1}
}
%}}}

\pagestyle{fancy}
\fancyfoot[c]{}
\fancyfoot[rE,lO]{Page \textbf{\thepage} of\textbf{~\pageref*{LastPage}}}


\pagenumbering{arabic}
\tcbuselibrary{external}
\tcbEXTERNALIZE


\title{A Long List of Problems for 33X}
\author{Jan Armendariz-Bones et al.}
\date{Last Updated: \today}

\begin{document}{\let\newpage\relax\maketitle}
\maketitle
The idea for this page came from a YouTube video called ``10,000'' problems in Analysis, which can be found \href{https://youtu.be/3mvNug_YM-g?si=9sa2EVkcANsY7IBk}{here}!

Of course, the goal is to compile a long list of problems ranging from very difficult to near trivial, all to get a better grasp of what it means to do \emph{real} advanced calculus problems.

Some shorthand on sources:

\noindent PMA = Principles of Mathematical Analysis (Rudin)

\noindent MSE = Mathematics StackExchange
\newpage
\begin{que}
		Source: Mathematics Discord \href{https://discord.com/channels/268882317391429632/576508782637744130/1169106442435969094}{(discord message)}

		By using the Cauchy Criterion for convergence, show that the sequence defined by \[\left\{x_n\right\}_1^\infty=\frac{1}{1^2}	+ \frac{1}{2^2}	+ \dots + \frac{1}{n^2}.\] converges.
\end{que}	

\begin{que}
		Source: Spring 1981 UC Berkley Mahtematics PhD Prelims, Question 16.

Let $f(x)$ be defined as a real-valued function for all $x\ge 1$, such that $f(1)=1$ and \[f^\prime(x) = \frac{1}{x^2 + (f(x))^2}.\]
Prove that \[\lim_{x\to\infty}f(x)\] exists and the limit is \emph{less than} $1+\frac{\pi}{4}$. 
\end{que}	
\begin{que}
Source: Real Analysis (Royden) Chapter 6.1 Question 2

Show that there exists a strictly increasing function $f(x)$ over the interval $[0,1]$, but $f(x)$ is continuous over only the irrationals in $[0,1]$ 
\end{que}	
\begin{que}
Source: PMA (Rudin) Chapter 5

Suppose $f^\prime(x)$ is continuous over an interval $[a,b]$ and let $\varepsilon>0$ . Prove that there exists some $\delta>0$ such that 
\[\left|\frac{f(t)-f(x)}{t-x}- f^\prime(x)\right|<\varepsilon\]
whenever $0<\left|t-s\right|<\delta$, $x\in [a,b]; y\in [a,b]$.

If this property holds, we say that $f$ is \emph{uniformly differentiable} on $[a,b]$.
\end{que}	
\begin{que}
Source: PMA (Rudin) Chapter 5

If $f(x) = \left|x\right|^3$, compute $f^\prime(x) \text{ and } f^{\prime\prime}(x)$  for all real $x$. Then show that $f^{(3)}(0)$ does not exist. 
\end{que}	
\begin{que}
		Source: MSE (\href{https://math.stackexchange.com/questions/4797989/determine-the-points-of-continuity-of-hx-lfloor-sinx-rfloor}{Question})

Determine the points of continuity of $h(x)=\lfloor \sin(x)\rfloor$ , where $\lfloor x \rfloor$  is the greatest $m\in\mathbb{Z}$ such that $m\le x$  (floor function).
\end{que}	
\begin{que}
		Source: Sample UC Davids Real Analysis Questions (\href{https://www.math.ucdavis.edu/~hunter/m125a/m125a_sample_final_solutions.pdf}{Has Solution})

		\begin{enumerate}[(a)]
				\item Suppose $f_n\colon A \to \mathbb{R}$ is \emph{uniformly continuous} on $A$ for every $n\in\mathbb{N}$ and $f_n\to f$ uniformly on $A$.
						Prove that $f$ is uniformly continuous on $A$. 
				\item Does the result in $(a)$ remain true if $f_n\to f$ pointwise instead of uniformly?
		\end{enumerate}	
\end{que}	
\begin{que}
Source: Problems in Real Analysis (R\v adlescu et. al) Chapter 5.2 No. 3

Let $f(x)$ be a polynomial and $a$ be a real number such that $f(a)\neq 0$. Show that there exists a polynomial with real coefficients $g(x)$ such that $p(a)=1$, $p^\prime(a)=0$, and $p^{\prime\prime}(a) = 0$, where $p(x)=f(x)g(x)$.  
\end{que}

\begin{que}
Source: Problems in Real Analysis (R\v alescu et. al) Chapter 5.2 No 8

Find all \smash{\underline{integers}} $a$ and $b$ such that $0<a<b$ and $a^b=b^a$. 
\end{que}	
\begin{que}
Source: Problems in Mathematical Analysis (Kazcor and Nowak) Exercise 2.2.29

Show that the only functions $f\colon\mathbb{R}\to \mathbb{R}$ such that \[\frac{f(x+h)-f(x)}{h}=f^\prime(x+\frac{h}{2})\] 

are polynomials of degree two.
\end{que}	
\begin{que}
Source: Problems in Mathematical Analysis (Kazcor and Nowak) Exercise 2.2.31

Prove that if $f$ is differentiable on an interval $\mathbf{I}$, then the Intermediate Value Theorem can be used on $\mathbf{I}$.

(This is referred to as \emph{Darboux's Theorem}.)
\end{que}
\begin{que}
Source: Kazcor and Nowak Exercise 2.3.9.

Let $f$  be an $n+1$ times differentiable function. Prove that for every $x\in \mathbb{R}$, there exists some $\theta\in(0,1)$ such that

\begin{enumerate}[(a)]
		\item \[f(x)=f(0)+xf^\prime(x) - \frac{x^2}{2}f^{\prime\prime}(x) +\dots + (-1)^{n+1}\frac{x^n}{n!}f^{(n)}(x) + (-1)^{n+2}\frac{x^{n+1}}{(n+1)!}f^{(n+1)}(\theta x).\]
		\item \[f\left(\frac{x}{1+x}\right) = f(x) - \frac{x^2}{1+x}f^\prime(x) + \dots + (-1)^n\frac{x^{2n}}{(1+x)^n}\frac{f^{(n)}(x)}{n!} + (-1)^{n+1}\frac{x^{2n+2}}{(1+x)^{n+1}}\frac{f^{(n+1)}(\frac{x+\theta x^2}{1+x})}{(n+1)!}\qquad x\neq -1.\]
\end{enumerate}	
\end{que}	
\begin{que}
		Source: MSE (\href{https://math.stackexchange.com/questions/4024737/if-a-two-variable-smooth-function-has-two-global-minima-will-it-necessarily-hav}{Source})

Suppose we have a function $f\colon\mathbb{R}^2\to\mathbb{R}$ such that $f$ is smooth ($\mathcal C^\infty$). Furthermore, suppose that $f$ has two global minima. Prove or disprove that $f$ has a third critical point. 
		\begin{itemize}
		\item What happens if we add the constraint that $f(x)\to\infty$ as $\|x\|\to\infty$?  
\end{itemize}	
\end{que}	

\begin{que}
Let $f,g\colon [0,1]\to \mathbb R$ be continuous functions such that for any continuous $\varphi\colon [0,1] \to \mathbb R$ where $\varphi(1)=\varphi(0)=0$, we have \[\int_0^1(f(x)\varphi^\prime(x) + g(x)\varphi(x))\:\mathrm dx=0\]

Prove that $f$ is continuous over it's domain and $f^\prime(x) = g(x)$.

This theorem is called the \emph{du-Bois Reymond Lemma}.
\end{que}	
\end{document}
